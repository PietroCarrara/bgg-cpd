\documentclass[hidelinks,12pt,article,twocolumn,brazil]{abntex2}

\usepackage{indentfirst}
\usepackage{graphicx}
\usepackage[utf8]{inputenc}
\usepackage[T1]{fontenc}
\usepackage[alf]{abntex2cite}
\usepackage[table,xcdraw]{xcolor}

\newcommand{\imagem}[3]{\begin{figure}[htb]
    \begin{center}
        \includegraphics[width=0.49\textwidth]{#1}\\
        \caption{#2}
        \label{#3}
    \end{center}
\end{figure}}

\newcommand{\mono}[1]{\texttt{#1}}

\def\bgg{\textit{boardgamegeek.com}}

\data{\today}
\autor{Pietro Benati Carrara}
\instituicao{Universidade Federal do Rio Grande do Sul}
\local{Porto Alegre}

\titulo{Implementação de Estruturas de Dados e Algoritmos Para Armazenamento e Recuperação de Informações de Jogos de Tabuleiro}

\begin{document}

\imprimircapa

\section{Visão Geral}

\subsection{Apresentação}

Jogos de tabuleiro são um importante elemento da cultura humana, tendo sido desenvolvidos e jogados na maior parte das sociedades ao longo do tempo \cite{livingstone2019board}. Com o advento da produção em massa e de mercados cada vez mais abrangentes, a produção de jogos de tabuleiro vem crescendo em ritmo acelerado \cite{quinns2017youtube}.

O grande número de jogos feitos por essa indústria em ascenção leva a necessidade de se catalogar e organizar essa produção, a fim de possibilitar a navegação deste crescente contingente. Este trabalho propõe uma ferramenta que realiza esta catalogação, e auxilia na busca por jogos ao cumprir os seguintes objetivos:
\begin{itemize}
    \item Categorizar e buscar jogos por gênero (estratégia, cartas, \ldots)
    \item Classificar e pesquisar jogos por mecânicas (rolagem de dados, tabuleiro modular, \ldots)
    \item Selecionar os resultados em ordem decrescente da avaliação da comunidade, mostrando primeiro jogos bem avaliados
    \item Classificar editoras que publicam os jogos, possibilitando a busca de empresas que operem próximas ao usuário
    \item Cadastrar avaliações do usuário sobre os jogos inseridos no sistema, habilitando o feedback da comunidade sobre jogos que gostaram ou não
\end{itemize}

\subsection{Conjunto de Dados Selecionados}

Os dados sobre jogos estão disponíveis em \bgg, e a ferramenta desenvolvida os extrai a partir de sua API pública. O site foi fundado em 2000 por Scott Alden \cite{woods2012eurogames}, e agrega informações sobre jogos, suas editoras e avaliações dos usuários.

Conhecido como ``a central para jogos de tabuleiro na internet'' \cite{draper2019bgg}, o site reúne usuários que formam uma comunidade ``onde seus membros são amplamente representativos do tipo de jogador com um forte interesse em jogos de tabuleiro'' \cite{woods2012eurogames}.

Dos dados disponíveis no site, a ferramenta desenvolvida extrai os jogos, assim como suas mecânicas, categorias, editoras\footnote{A API disponibiliza vários dados, mas não editoras; para obter as informações necessárias usou-se a biblioteca \href{https://www.crummy.com/software/BeautifulSoup/}{BeautifulSoup}, que intepreta a própria interface HTML do site.}, expansões e avaliações da comunidade.

\section{Implementação}

\subsection{Diagrama Entidade--Relacionamento}

Ao criar uma abstração para representar os dados utilizados pela ferramenta, escolheu-se o modelo Entidade--Relacionamento, que descreve dados em termos de entidades, suas informações e como se relacionam umas com as outras \cite{chen1976entity}.

\imagem{img/ER-CPD.png}{Diagrama entidade--relacionamento modelando o conjunto de dados}{diagrama_er}

Para a conversão do modelo abstrato mostrado na figura \ref{diagrama_er} em uma série de arquivos relacionais, as informações foram transformadas para a 3ª forma normal -- que tem como objetivo ``possibilitar a pesquisa eficiente através de uma coleção mais simples de operações relacionais'' \cite{codd1971normalized} -- através do seguinte procedimento:

\begin{enumerate}
    \item Transformar cada entidade em uma tabela
    \item As relações 1--N se transformam em um atributo da entidade de cardinalidade N, e aponta para a entidade de cardinalidade 1
    \item As relações N--N se transformam em suas próprias tabelas, onde cada tupla armazena os identificadores das tabelas envolvidas na relação
\end{enumerate}

\subsection{Estruturas de Dados}

Para armazenamento dos dados em disco, foram utilizadas estruturas de dados mais adequadas a este meio de armazenamento.

\subsubsection{Tabelas e Índices}
\label{estrut_tabelas}

As tabelas são armazenadas em arquivos sequenciais, sem critério de ordem. Apesar disso, seu acesso é \textit{randômico}, significando que pode-se acessar rápidamente qualquer elemento, contanto que se conheça sua localização. Para a busca rápida baseada no identificador de uma entidade, seus índices são armazenados em BTrees: árvores auto-balanceáveis mais adequadas para armazenamento em memória secundária, já que conseguem localizar dados com poucos \textit{disk reads} \cite{knuth1973art}.

\subsection{Índices Reversos}
\label{estrut_indices}

Para rápida localização de elementos baseado no valor de seus atributos, foram implementados índices reversos. Essa estrutura está dividida em duas partes: um arquivo de dicionário e um arquivo de postings.

A estrutura de postings armazena várias listas em blocos contendo índices que apontam para uma tabela. Pelo fato de ser uma lista em blocos, é mais adequada ao armazenamento em disco, já que este meio dá vantagem à leitura em blocos.

O dicionário é chaveado pelos valores de alguma propriedade de uma tabela, e armazena um ``ponteiro'' para a lista no arquivo de postings que cotém os índices de todas as entradas que possuem aquele dado valor em sua propriedade.

\subsection{Estrutura do Código}

Nenhuma das estruturas mencionadas nas seções \ref{estrut_tabelas} e \ref{estrut_indices} assume que tipo de dado irão armazenar. Esta função é delegada ao objeto chamado de \texttt{persist}. É o \texttt{persist} que sabe como codificar um jogo, expansão, comentário, etc\ldots para bytes, e de bytes de volta para seu tipo original. Isso possibilita a alta reutilização de código das estruturas.

Todas essas estruturas são gerenciadas pela classe \texttt{Database}, que abstrai seu uso para métodos mais simples, como \texttt{get\_by\_id} ou \texttt{get\_by\_posting}.

Para a exibição dos dados e interação com o usuário, foi contruída uma interface textual para o \textit{console}, utilizando a biblioteca \href{https://github.com/jwlodek/py_cui}{\texttt{py\_cui}}.

Para possibilitar a busca textual de editoras , jogos, categorias e mecânicas, seus nomes e descrições foram \textit{tokenizados} e indexados utilizando índices reversos.

\subsection{Arquivos De Armazenamento}

Ao total, a base utiliza 41 arquivos, cada um nomes, extensões, informações e estruturas conforme descritas a seguir:

\begin{itemize}
    \item \textbf{Tabelas Sequenciais (.table)}: \mono{categories}, armazenando categorias; \mono{comments}, armazenando comentários; \mono{expansions}, armazenando expansões; \mono{game\_category}, armazenando tuplas de (jogo, categoria); \mono{game\_mechanic}, armazenando tuplas de (jogo, mecânica); \mono{game\_publisher}, armazenando tuplas de (jogo, editora); \mono{games}, armazenando jogos; \mono{mechanics}, armazenando mecânicas; \mono{publishers}, armazenando editoras.

    \item \textbf{Índices em BTrees (.btree)}: \mono{categories} indexando categorias; \mono{comments} indexando comentários; \mono{expansions} indexando expansões; \mono{games} indexando jogos; \mono{mechanics} indexando mecânicas; \mono{publishers} indexando editoras.

    \item \textbf{Índices Reversos (.dictionary e .posting)}: \mono{categories\_word} indexando textualmente categorias; \mono{comments\_expansion} indexando comentários que comentam uma dada expansão; \mono{comments\_game} indexando comentários que comentam um dado jogo; \mono{expansions\_game} expansões que expandem um dado jogo; \mono{game\_category\_category} indexando a relação entre jogos e categorias, chaveado por categoria; \mono{game\_category\_game} indexando a relação entre jogos e categorias, chaveado por jogo; \mono{game\_mechanic\_game} indexando a relação entre jogos e mecânicas chaveado por jogo; \mono{game\_mechanic\_mechanic} indexando a relação entre jogos e mecânicas, chaveado por mecânica; \mono{game\_publisher\_game} indexando a relação entre jogos e editoras, chaveado por jogo; \mono{game\_publisher\_publisher} indexando a relação entre jogos e editoras, chaveado por editora; \mono{games\_word} indexando textualmente jogos; \mono{mechanics\_word} indexando textualmente mecânicas; \mono{publishers\_word} indexando textualmente editoras.
\end{itemize}

\subsection{Algoritmos Desenvolvidos e Seus Usos}

\section{Guia de Uso}

Ao iniciar o programa, rodando o comando \texttt{\$ python main.py}, o usuário se encontra na tela inicial (figura \ref{scr_init}). O foco pode ser movido com as setas do teclado. Esta tela apresenta 3 opções: Busca de jogos, busca de editoras e preencher a base de dados.

\imagem{img/scr_init.png}{Tela Inicial}{scr_init}

Caso o usuário selecione a 3ª opção, um \textit{pop-up} será exibido (figura \ref{scr_fill}), informando o progresso do programa conforme preenche a base de dados com informações extraídas de \bgg.

\imagem{img/scr_fill}{Tela de Preenchimento da Base de Dados}{scr_fill}

Caso o usuário selecione a opção de busca de jogos, será levado à tela vista na figura \ref{scr_search}. Aqui é possível buscar mecânicas/categorias e selecioná-las para que sejam exibidos somente jogos que usem essas mecânicas e sejam dessas categorias. É também possível informar um texto, para que sejam selecionados jogos onde as palavras informadas apareçam em sua descrição ou título.

\imagem{img/scr_search.png}{Tela de Busca de Jogos}{scr_search}

Ao selecionar um jogo da tela de busca, o usuário é levado para a tela do jogo (figura \ref{scr_game}). Aqui é possível ver as informações do jogo: Número de jogadores, tempo de jogo, suas expanções, comentários, categorias, mecânicas, etc\ldots Caso o usuário selecione alguma categoria ou mecânica, será levado novamente para a tela de busca, mas exibindo jogos daquela mecânica/gênero. Caso o usuário selecione uma das editoras que publicam o jogo, ele será levado para a tela da editora (figura \ref{scr_publisher}).

\imagem{img/scr_game.png}{Tela do Jogo \textit{Galaxy Trucker}}{scr_game}

Na tela de expansão, podemos ver as informações da expansão, como seu nome, descrição e comentários. Para voltar à tela anterior, o usuário pode pressionar a tecla \texttt{ESC} para sair do foco de qualquer widget, e \texttt{Q} para sair da tela atual. Caso faça isto na tela inicial, o programa é encerrado.

\imagem{img/scr_expansion.png}{Tela da expansão \textit{The Pillbug} para o jogo \textit{Hive}}{scr_scr_expansion}

Se, a partir da tela inicial, o usuário selecionar a busca de editoras, será levado para a tela mostrada na figura \ref{scr_searchpub}. Aqui é possível informar um texto para a busca, que tentará encontrar editoras que contenham o texto em seu título ou descrição.

\imagem{img/scr_searchpub.png}{Tela de Busca de Editoras}{scr_searchpub}

Ao selecionar uma editora, o usuário é levado à tela da editora (figura \ref{scr_publisher}), onde é possível consultar seu nome, descrição e jogos publicados.

\imagem{img/scr_publisher.png}{Tela da Editora ``Galápagos Jogos''}{scr_publisher}

\section{Contribuição}

Para analisar a contribuição deste trabalho para com a literatura, elaborou-se uma tabela com quesitos comparando a ferramenta aqui proposta com a maior referência na área, {\bgg} (BGG).

\begin{table}[htb]
    \rowcolors{1}{gray!15}{gray!5}
    \centering
    \begin{tabular}{ccc}
        Quesito                                                                     & \begin{tabular}[c]{@{}c@{}}Ferramenta\\ Proposta\end{tabular} & BGG                         \\
        \begin{tabular}[c]{@{}c@{}}Possui\\ busca\\ avançada\end{tabular}           & \cellcolor[HTML]{9AFF99}Sim                                   & \cellcolor[HTML]{9AFF99}Sim \\
        \begin{tabular}[c]{@{}c@{}}Indexa\\ jogos\\ e suas\\ expansões\end{tabular} & \cellcolor[HTML]{9AFF99}Sim                                   & \cellcolor[HTML]{9AFF99}Sim \\
        \begin{tabular}[c]{@{}c@{}}Interface\\ acessível\end{tabular}               & \cellcolor[HTML]{FD6864}Não                                   & \cellcolor[HTML]{9AFF99}Sim \\
        \begin{tabular}[c]{@{}c@{}}Disponivel\\ offline\end{tabular}                & \cellcolor[HTML]{9AFF99}Sim                                   & \cellcolor[HTML]{FD6864}Não
    \end{tabular}
    \caption{Comparação entre funcionalidades das ferramentas}
    \label{tab_comp_func}
\end{table}

Como pode-se ver na tabela \ref{tab_comp_func}, ambas as ferramentas apresentam buscas avançadas, assim como indexam não apenas jogos, mas suas expansões. Percebe-se também que a ferramenta proposta apresenta uma interface menos amigável do que \bgg. Apesar disso, pelo fato de que {\bgg} é um sistema web, o usuário precisa ter uma conexão com a internet para fazer uso desse, enquanto que a ferramenta proposta precisa de internet apenas durante a carga de dados.

\bibliography{report}

\end{document}