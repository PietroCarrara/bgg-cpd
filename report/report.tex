\documentclass[hidelinks,article,twocolumn,brazil]{abntex2}

\usepackage{indentfirst}
\usepackage[utf8]{inputenc}
\usepackage[T1]{fontenc}
\usepackage[alf]{abntex2cite}

\def\bgg{\textit{boardgamegeek.com}}

\data{\today}
\autor{Pietro Benati Carrara}
\instituicao{Universidade Federal do Rio Grande do Sul}
\local{Porto Alegre}

\titulo{Implementação de Estruturas de Dados e Algoritmos Para Armazenamento e Recuperação de Informações de Jogos de Tabuleiro}

\begin{document}

\imprimircapa

\section{Visão Geral}

\subsection{Apresentação}

Jogos de tabuleiro são um importante elemento da cultura humana, tendo sido desenvolvidos e jogados na maior parte das sociedades ao longo do tempo \cite{livingstone2019board}. Com o advento da produção em massa e de mercados cada vez mais abrangentes, a produção de jogos de tabuleiro vem crescendo em ritmo acelerado \cite{quinns2012youtube}.

O grande número de jogos feitos por essa indústria em ascenção leva a necessidade de se catalogar e organizar essa produção, a fim de possibilitar a navegação deste grande contingente. Este trabalho propõe uma ferramenta que realiza esta catalogação, e auxilia na busca por jogos através dos seguintes objetivos:
\begin{itemize}
    \item Categorização e busca de jogos por gênero (estratégia, cartas, \ldots)
    \item Classificação e pesquisa de jogos por mecânicas (rolagem de dados, tabuleiro modular, \ldots)
    \item Ordenamento dos resultados em ordem decrescente da avaliação da comunidade
    \item Classificação das editoras que publicam os jogos, possibilitando a busca de empresas que operem próximas ao usuário
    \item Cadastro da avaliação do usuário dos jogos inseridos no sistema, possibilitando o feedback da comunidade sobre jogos que gostaram ou não
\end{itemize}

\subsection{Conjunto de Dados Selecionados}

Os dados sobre jogos estão disponíveis em \bgg, e a ferramenta desenvolvida os extrai a partir de sua API pública. O site foi fundado em 2000 por Scott Alden \cite{woods2012eurogames}, e agrega informações sobre jogos, suas editoras e avaliações dos usuários.

Conhecido como ``a central para jogos de tabuleiro na internet'' \cite{draper2019bgg}, o site reúne usuários que formam uma comunidade ``onde seus membros são amplamente representativos do tipo de jogador com um forte interesse em jogos de tabuleiro'' \cite{woods2012eurogames}.

Dos dados disponíveis no site, a ferramenta desenvolvida extrai os jogos, assim como suas mecânicas, categorias, editoras, expansões e avaliações da comunidade.

\section{Implementação}

\subsection{Estrutura do Código}

\subsection{Estruturas de Dados}

\subsection{Arquivos Utilizados}

\subsection{Algoritmos Desenvolvidos e Seus Usos}

\section{Guia de Uso}

\section{Contribuição}

\bibliography{report}

\end{document}